% Chapter 1

\chapter{Introduction} % Main chapter title

\label{Chapter1} % For referencing the chapter elsewhere, use \ref{Chapter1}

%----------------------------------------------------------------------------------------

% Define some commands to keep the formatting separated from the content
\newcommand{\keyword}[1]{\textbf{#1}}
\newcommand{\tabhead}[1]{\textbf{#1}}
\newcommand{\code}[1]{\texttt{#1}}
\newcommand{\file}[1]{\texttt{\bfseries#1}}
\newcommand{\option}[1]{\texttt{\itshape#1}}

%----------------------------------------------------------------------------------------

\section{Introduction}



% 1 = \cite{Beni2004}
% 2 = \cite{BongardZykovLipson2006}
% 3 = \cite{BrooksFlynn1989}
% 5 = \cite{CullyCluneTaraporeMouret2015}
% 8 = \cite{DoncieuxBredecheMouretEiben2015}
% 10 = \cite{fenton2001fault}
% 21 = \cite{verma2004real}

%%%%%%%%%%%%%%%%%%%%%%%%%%%%%%%%%%%%%%%%%%%%%%%%%%%%%%%%%%%%%%%%%%%%%%%%%%%%%%%%%%%%%%%%
%%	"Autonomous robots are increasingly being applied to remote and hazardous environ	%%
%%	ments [3,1], environments damage to sensory actuator systems cannot be easily rep	%%
%%	aired if damaged. An unsolved problem in the controller design for such autonomou	%%
%%	s robots is having controllers continue to effectively function given unexpected 	%%
%%	changes such as damage to robot morphology."                                    	%%
%%	                                                                                	%%
%%	"Currently, robotic systems recover from damage via self diagnosis and selection 	%%
%%	from pre-designed contingency plans in order to continue functioning [10,21,2]. T	%%
%%	hough robots using such self-diagnosis and recovery are problematic systems as th	%%
%%	ey are expensive, requiring sophisticated monitoring sensors, and difficult to de	%%
%%	sign as a priori knowledge of all necessary contingency plans is assumed [5]""  	%%
%%	                                                                                	%%
%%	"Addressing this, recent work in Evolutionary Robotics [8] elucidated the efficac	%%
%%	y of population based stochastic trial and error methods for online damage recove	%%
%%	ry in autonomous robots operating in physical environments [5]. This was demonstr	%%
%%	ated as being akin to self-adaptation and injury recovery of animals observed in 	%%
%%	nature.""                                                                       	%%
%%	                                                                                	%%
%%	"This study further contributes to this research area, focussing on evolutionary 	%%
%%	controller design [11] within the broader context of collective [15] and swarm [1	%%
%%	] robotics. That is, evolutionary controller design for robot groups that must co	%%
%%	ntinue to accomplish tasks given that damage is sustained to the morphologies of 	%%
%%	some or all of the robots in the group. ""
%%%%%%%%%%%%%%%%%%%%%%%%%%%%%%%%%%%%%%%%%%%%%%%%%%%%%%%%%%%%%%%%%%%%%%%%%%%%%%%%%%%%%%%%




% ====================================================
The design and development of automated problem solvers and autonomous robots has long been a central theme of research in the crossover between artificial intelligence and robotics \cite{RefWorks:33}. However, given the levels of complexity and non-linearity of real-world problems, traditional methods of controller design have become impractical since they rely mostly on linear mathematical models \cite{RefWorks:32}. An additional consideration is that at a certain point, designing the problem solver becomes more difficult than solving the problem, especially when there is not a distinct and well-defined set of steps to follow in order to obtain the solution.

One such complex task, and the one being focussed on as the subject of this thesis, is that of achieving automated collective behaviour in which a team of autonomous agents must cooperate amongst themselves in order to accomplish a task. More specifically, this thesis will focus on collective robotics with a case study on the collective construction task

While it may be possible to to solve many complex tasks using a single sophisticated robot, there are several advantages to using collective robotics instead. Some of these advantages include being able to use much simpler designed robots in the team instead of one very expensive robot as well as having increased fault tolerance since there is no longer a single point of failure \cite{chaimowicz2001architecture}. The downside to a multi-agent team, however, is the infeasibility of designing the intricate dynamics inherent in producing self-organised behaviour between individuals by hand \cite{RefWorks:11}.

The collective construction task requires that a team of agents cooperate by coordinating their behaviours in order to assemble various structures within their environment \cite{NitschkeSaEC2012}. A group of autonomous agents (robots) are deployed into an unknown environment which they must explore in order to find resources and figure out how to assemble these resources in such a way as to create some structure, which can be completely predefined, or just according to some predefined structural criteria \cite{NitschkeSaEC2012}.

The ability to automate construction tasks through creating controllers for collective robotics presents several considerable benefits. Automation like this could be used to assist in projects such as producing low-cost housing in locations where resources are scarce, reducing high accident rates due to human error which is unfortunately unavoidable in traditional construction methods \cite{ShenKhoshnevis2003}. An additional possible application in the near future is to send multi-robot construction teams to extraterrestrial planets in order to build habitable structures in anticipation of humans arriving. These robot teams would also be useful in underwater construction which is extremely difficult and is one of the most dangerous jobs \cite{RefWorks:30}.

In an attempt to try and find a solution around the complexity of automating the design of these controllers, researchers turned to biological and natural processes for inspiration.

The problem solving power of evolution is abundantly clear in nature with the diverse range of organisms that exist (or ever have existed) on Earth, each specially designed for optimal survival in its own niche environment \cite{RefWorks:33}, making it unsurprising that computer scientists have resorted to researching natural processes for inspiration. 

One such approach that draws inspiration from Darwinian evolution is Evolutionary Algorithms; which relates the fundamental metaphor of the evolutionary process to a simple style of problem solving, which is that of trial-and-error (also referred to as generate-and-test) \cite{RefWorks:33}.

%%%%%%%%%%%%%%%%%%%%%%%%%%%%%%%%%%%%%%%%%%%%%%%%%%%%%%%%%%%%%%%%%%%%%%%%%%%%%%%%%%%%%%%%%%
%%	Whereas traditional/standard learning...                                        	%%
%%	NE implements the driving forces of natural selection (survival of the fittest/se	%%
%%	lection, reproduction/crossover, random mutations) in analagous processes so as t	%%
%%	o be able to modify the topology and connection weights between nodes to such a p	%%
%%	oint that a desired output behaviour is achieved.
%%%%%%%%%%%%%%%%%%%%%%%%%%%%%%%%%%%%%%%%%%%%%%%%%%%%%%%%%%%%%%%%%%%%%%%%%%%%%%%%%%%%%%%%%%


%%%%%%%%%%%%%%%%%%%%%%%%%%%%%%%%%%%%%%%%%%%%%%%%%%%%%%%%%%%%%%%%%%%%%%%%%%%%%%%%%%%%%%%%%%
%%	When compared with traditional learning in ANN's, such as Backpropagation and/or 	%%
%%	Gradient Descent methods where there is a target configuration or at least some e	%%
%%	rror can be calculated, Evolutionary Algorithms work in problem spaces where the 	%%
%%	desired structure is not directly known.
%%%%%%%%%%%%%%%%%%%%%%%%%%%%%%%%%%%%%%%%%%%%%%%%%%%%%%%%%%%%%%%%%%%%%%%%%%%%%%%%%%%%%%%%%%

%%%%%%%%%%%%%%%%%%%%%%%%%%%%%%%%%%%%%%%%%%%%%%%%%%%%%%%%%%%%%%%%%%%%%%%%%%%%%%%%%%%%%%%%%%
%%	Neuroevolution (NE) is a method of artificially evolving Artificial Neural Network	%%
%%	s (ANN's) by implementing Evolutionary Algorithms to perform tasks such as evolvin	%%
%%	g the network connection weights as well as the network topology \cite{Stanley2004	%%
%%	, XinYao1999}.
%%%%%%%%%%%%%%%%%%%%%%%%%%%%%%%%%%%%%%%%%%%%%%%%%%%%%%%%%%%%%%%%%%%%%%%%%%%%%%%%%%%%%%%%%%


Neuroevolution (NE) provides a way of combining EAs and ANNs \cite{RefWorks:31}. It uses EAs to evolve the connection weights, topologies and activation functions of ANNs in order to learn a specific task or behaviour \cite{gomez1999solving}. NE searches for an ANN with optimal performance based on a fitness function that determines an ANN's suitability to performing a task \cite{RefWorks:31}.


The process of natural selection/development is abstracted by using CPPN's to encode ANNs and allows for the production of complex patterns by determining the attributes of their phenotypes as functions of their geometric location \cite{clune2011performance}.

%%%%%%%%%%%%%%%%%%%%%%%%%%%%%%%%%%%%%%%%%%%%%%%%%%%%%%%%%%%%%%%%%%%%%%%%%%%%%%%%%%%%%%%%%%
%%	"Neuroevolution allows combining evolution over a population of solutions with li	%%
%%	fetime learning in individual solutions: the evolved networks can each learn furt	%%
%%	her through eg. Backpropagation or Hebbian learning." \cite{Miikkulainen2010}.
%%%%%%%%%%%%%%%%%%%%%%%%%%%%%%%%%%%%%%%%%%%%%%%%%%%%%%%%%%%%%%%%%%%%%%%%%%%%%%%%%%%%%%%%%%


It makes use of Evolutionary Algorithms in order to abstract the key aspects of the evolutionary process into a generational loop that will search for an optimal solution out of a population of candidates by evaluating each one in order to determine which is the fittest and thus the characteristics that persist to the next generation. Resulting in a stochastic search for the most optimal solution of the problem space.

This generate-and-test approach is analogous to natural selection and survival of the fittest, in this case, the characteristics most suitable to the environment are defined by the experimenters.

Extending this biological metaphor, we look at various other aspects of nature to further improve the abilities of 

%%%%%%%%%%%%%%%%%%%%%%%%%%%%%%%%%%%%%%%%%%%%%%%%%%%%%%%%%%%%%%%%%%%%%%%%%%%%%%%%%%%%%%%%%%
%%	somehow relate the evolutionary metaphor to the reason for using Novelty search as	%%
%%	well
%%%%%%%%%%%%%%%%%%%%%%%%%%%%%%%%%%%%%%%%%%%%%%%%%%%%%%%%%%%%%%%%%%%%%%%%%%%%%%%%%%%%%%%%%%


A common and well-known problem that is often used as a benchmark test for automated controllers is the pole-balancing problem \cite{Stanley2004}.


% \hl{"For the time being, let us consider natural evolution simply as follows. A given environment is filled with a population of individuals that strive for survival and reproduction. The fitness of these individuals - determined by the environment - relates to how well they succeed in achieving their goals, i.e., it represents their chances of survival and multiplying."} \cite{EibenSmith2003}

% \hl{"Natural selection favours those individuals that compete for the given resources most effectively, in other words, those that are adapted or fit to the environmental conditions best. This phenomenon is also known as Survival Of The Fittest."} \cite{EibenSmith2003}




%%%%%%%%%%%%%%%%%%%%%%%%%%%%%%%%%%%%%%%%%%%%%%%%%%%%%%%%%%%%%%%%%%%%%%%%%%%%%%%%%%%%%%%%
%%	\hl{"Competition-based selection is the one of the two cornerstones of evolutiona	%%
%%	ry progress. The other primary force identified by Darwin results from phenotypic	%%
%%	variations among members of the population. Phenotypic traits aare those behavio	%%
%%	ural and physical features of an individual that directly affect its response to 	%%
%%	the environment (including other individuals), thus determining its fitness."} \c	%%
%%	ite{EibenSmith2003}
%%%%%%%%%%%%%%%%%%%%%%%%%%%%%%%%%%%%%%%%%%%%%%%%%%%%%%%%%%%%%%%%%%%%%%%%%%%%%%%%%%%%%%%%



%%%%%%%%%%%%%%%%%%%%%%%%%%%%%%%%%%%%%%%%%%%%%%%%%%%%%%%%%%%%%%%%%%%%%%%%%%%%%%%%%%%%%%%%%%
%%	\hl{"Accidents happen in nature from simple incidents like bumping into obstacles,	%%
%%	to erroneously arriving at the wrong location, to mating with an unintended partn	%%
%%	er. Whether accidents are problematic for an animal depends on their context, freq	%%
%%	uency, and severity."} \cite{ferreira2018accidental}.
%%%%%%%%%%%%%%%%%%%%%%%%%%%%%%%%%%%%%%%%%%%%%%%%%%%%%%%%%%%%%%%%%%%%%%%%%%%%%%%%%%%%%%%%%%




%%%%%%%%%%%%%%%%%%%%%%%%%%%%%%%%%%%%%%%%%%%%%%%%%%%%%%%%%%%%%%%%%%%%%%%%%%%%%%%%%%%%%%%%%%
%%	\hl{"In this paper, we investigate the question of how accidents affect the task p	%%
%%	erformance of agents in an agent-based simulation modelf for a wide class of tasks	%%
%%	call 'Multi-Agent Territory Exploration' (MATE). In MATE tasks, agents have to vi	%%
%%	sit particular locations of varying quality in partially observable environments w	%%
%%	ithin a fixed time window. As such, agents have to balance the quality of the loca	%%
%%	tion with how much energy they are willing to expend reaching it. Arriving at the 	%%
%%	wrong location by acccident, typically reduces task performance."} \cite{ferreira2	%%
%%	018accidental}.
%%%%%%%%%%%%%%%%%%%%%%%%%%%%%%%%%%%%%%%%%%%%%%%%%%%%%%%%%%%%%%%%%%%%%%%%%%%%%%%%%%%%%%%%%%


%%%%%%%%%%%%%%%%%%%%%%%%%%%%%%%%%%%%%%%%%%%%%%%%%%%%%%%%%%%%%%%%%%%%%%%%%%%%%%%%%%%%%%%%%%
%%	\hl{"We model agents based on two location selection strategies that are hypothesi	%%
%%	zed to be widely used in nature: best-of-n and min-threshold. Our results show tha	%%
%%	t the two strategies lead to different accident rates and thus overall different l	%%
%%	evels of performance based on the degree of competition among agents, as well as t	%%
%%	he quality, density, visibility, and distribution of target locations in the envir	%%
%%	onment. We also show that in some cases individuual accidents can  be advantageous	%%
%%	for noth the individual and the whole group"} \cite{ferreira2018accidental}
%%%%%%%%%%%%%%%%%%%%%%%%%%%%%%%%%%%%%%%%%%%%%%%%%%%%%%%%%%%%%%%%%%%%%%%%%%%%%%%%%%%%%%%%%%




%% The below can be used somewhere like Future Work or Considerations in the Experiments:
%%%%%%%%%%%%%%%%%%%%%%%%%%%%%%%%%%%%%%%%%%%%%%%%%%%%%%%%%%%%%%%%%%%%%%%%%%%%%%%%%%%%%%%%%%
%%	\hl{"Evolutionary processes can be simulated in a computer, where millions of gene	%%
%%	rations can be executed in a matter of hours or days and repeated under various ci	%%
%%	rcumstances. These possibilities go far beyond studies based on excavations and fo	%%
%%	ssils, or those possible 'in vivo'. Naturally, the interpretation of such simulati	%%
%%	on experiments must be done very carefully. First, because we do not know whether 	%%
%%	the computer models represent the biological reality with sufficient fidelity. Sec	%%
%%	ond, it is unclear whether conclusions drawn in a digital medium, 'in silico', can	%%
%%	be transferred to the carbon-based biological medium. These caveats and the lack 	%%
%%	of mutual awareness between biologists and computer scientists are probably the re	%%
%%	ason why there are few computer experimental studies about fundamental issues of b	%%
%%	iological evolution"} \cite{EibenSmith2003}.
%%%%%%%%%%%%%%%%%%%%%%%%%%%%%%%%%%%%%%%%%%%%%%%%%%%%%%%%%%%%%%%%%%%%%%%%%%%%%%%%%%%%%%%%%%




% ===============================================================
% EDIT 1:
% -----------------------
% ADD SOME REAL WORLD EXAMPLES AND DESCRIPTIONS OF NE 
% "NE is the main method for controller adaptation in evolutionary robotics (the overall topic that this thesis fits into) - and within this field - NE applied to collective robotics has achieved success (evolved effective group behaviours) to solve various collective behaviour tasks - then give several example tasks that evolutionary collective robotics systems have solved - with references"
% ==================================================================


Given that there are several learning tasks in the real world, such as game playing, vehicle control, and robotics, that are not well-suited to supervised learning methods, Neuroevolution aims to find an ANN that optimizes behaviour given only sparse information regarding how \textit{well} the agents are performing, and without any input on \textit{what} they should be doing \cite{Miikkulainen2010}. 

In the above outlined scenarios, the optimal action at each timestep is not always known and can only be determined after performing several actions to get information about the outcome of those actions and whether or not it was beneficial given the defined parameters \cite{Miikkulainen2010}.

This makes Neuroevolution a suitable approach for this dissertation as it falls within the specialised problem-space.

%%%%%%%%%%%%%%%%%%%%%%%%%%%%%%%%%%%%%%%%%%%%%%%%%%%%%%%%%%%%%%%%%%%%%%%%%%%%%%%%%%%%%%%%%%
%%	\hl{"Neuroevolution is a combination of neural networks and Genetic Algorithms whe	%%
%%	re neural networks are the phenotype being evaluated"} \cite{Stanley2004}.      	%%
%%	                                                                                	%%
%%	\hl{"The genotype is a compact representation that can be translated into an artif	%%
%%	icial neural network. NE searches for neural networks that optimize some performan	%%
%%	ce measure. NE can search for virtually any kind of neural network whether it be s	%%
%%	imple, feed-forward, recurrent, or even adaptive networks"} \cite{Stanley2004}. 	%%
%%	                                                                                	%%
%%	When compared to various other neural network learning methods, NE can be observed	%%
%%	to be highly general. This allows for learning to be performed without explicit t	%%
%%	argets, with non-differentiable activation functions, as well as recurrent network	%%
%%	s \cite{Miikkulainen2010}.                                                      	%%
%%	                                                                                	%%
%%	\hl{"As long as the performance of the networks can be evaluated over time, and th	%%
%%	e behaviour of the network can be modified through evolution, it can be applied to	%%
%%	a wide range of network architectures, including those with nondifferentiable act	%%
%%	ivation functions and recurrent and higher-order connections"} \cite{Miikkulainen2	%%
%%	010}.
%%%%%%%%%%%%%%%%%%%%%%%%%%%%%%%%%%%%%%%%%%%%%%%%%%%%%%%%%%%%%%%%%%%%%%%%%%%%%%%%%%%%%%%%%%




The different types of NE methods can be roughly broken down into two different categories, with respect to the encoding method used to map an individual agent's genotype to its phenotype %(from PTTAND010_Literature_Review.pdf)


The level of regularity that an EA tends to produce is affected by the encoding, which is how the information is stored in the genome and the process by which that information produces the phenotype. 
Regularity in evolved solutions is less likely to emerge with direct encodings, wherein each element in the genotype encodes an independent aspect of the phenotype.
In contrast, regularities are common with indirect encodings (also known as generative or developmental encodings), wherein information in the genome can be reused to affect many parts of the phenotype


%% \hl{Direct Encoding: NEAT}

In direct encoding methods, each element in the genotype explicitly encodes an independent aspect of the phenotype, such that there is a one-to-one mapping between the genotype and the phenotype \cite{clune2011performance,stanley2009hypercube}.

In this encoding scheme, all the parameters of an ANN's architecture are explicitly encoded on the chromosome as binary strings \cite{Gomez2003}. Each gene in the chromosome directly relates to a specific part of the network, allowing for a direct one-to-one mapping from the genotype to the phenotype \cite{StanleyMiikkulainen2002}.

The direct encoding scheme is relatively straight-forward to implement and it allows for a single connection to be easily added or removed from an ANN, making it suitable for precise fine-tuning of an architecture \cite{StanleyMiikkulainen2002}.
It may also accommodate for optimization and generation of unexplored architectures \cite{miller1989designing}.

On the other hand, a major disadvantage of direct encoding schemes is that it does not scale well \cite{XinYao1999}.  The reason for this is that the components of the network are mapped directly onto the chromosome, meaning there is no compression of information and would become computationally expensive as the size of the network grows.

An example of one such algorithm that makes use of a direct encoding method, is the Neuroevolution of Augmenting Topologies algorithm, also referred to as NEAT \cite{Gomez2003}.

NEAT is initialised with a small population of simple networks which are then gradually made more complex by either adding, removing, or mutating network connections \cite{RefWorks:11,Gomez2003}.


%% \hl{Indirect Encoding: HyperNEAT}

This encoding scheme functions at a higher level of abstraction than direct encoding  \cite{Gomez2003}.

An example implementation method of this encoding scheme includes using parametric functions to represent genes in a chromosome and using predetermined developmental rules to construct architectures \cite{koutnik2010evolving}.

In indirect encoding (also referred to as generative or developmental encoding) methods, each element in the genotype is implicitly described by computable function allowing for a much more compact representation of the genotype \cite{clune2011performance,stanley2009hypercube}. It also allows for genetic information to be re-used.

In other words, this algorithm transforms (decodes) the genotype to the phenotype by using any computable function \cite{koutnik2010evolving}.


It has been shown that NE can successfully be applied to complex control tasks that require collective behaviour. These control tasks have no obvious mappings between the input from sensory configurations and motor outputs \cite{NitschkeSaEC2012}.


%% Explanation of HyperNEAT algorthim
Instead of being directly applied to ANN's, HyperNEAT works by using the NEAT algorithm (Neuroevolution of Augmetnting Topologies) to evolve a CPPN (Compositional Pattern Producing Network \cite{StanleyMiikkulainen2002}), which is a genotype (network) representing the connections of a possibly much larger network \cite{RefWorks:14}.

%%The topics mentioned above are discussed in greater detail later on in this paper \hl{reference to the appropriate chapter}.




%%%%%%%%%%%%%%%%%%%% EXAMPLE APPLICATIONS REAL WORLD %%%%%%%%%%%%%%%%





%% ====== Motivation for investigating Novelty Search can be explained as being analagous to accidental encounters being beneficial or adaptive =======

%%%%%%%%%%%%%%%%%%%%%%%%%%%%%%%%%%%%%%%%%%%%%%%%%%%%%%%%%%%%%%%%%%%%%%%%%%%%%%%%%%%%%%%%%%
%%	\hl{"Take, for example, the common instance of a MATE task where agents (eg. femal	%%
%%	es) attempt to find (stationary) mates located in the environment using different 	%%
%%	mate selection strategies based on males's mating calls or displays of prowess, wh	%%
%%	ich are indicative of the quality of the mate and thus a crucial determinant of th	%%
%%	e quality of the offspring. Since females typically only have partial knowledge of	%%
%%	the location of possible partners, as some males mayt not advertise their locatio	%%
%%	n through calls, they may end up mating with non-optimal partners by accident (eg.	%%
%%	bumping into a potential mate causes females to simply mate with the male in some	%%
%%	species such as tree frogs). For low quality males who might not win in a 'shouti	%%
%%	ng' competition with high quality males, it might thus be beneficial to remain sil	%%
%%	ent and try to intercept females instead of calling, a strategy often referred to 	%%
%%	as 'satellite' strategy; for if they decide to call, females might actively avoid 	%%
%%	them."} \cite{ferreira2018accidental}
%%%%%%%%%%%%%%%%%%%%%%%%%%%%%%%%%%%%%%%%%%%%%%%%%%%%%%%%%%%%%%%%%%%%%%%%%%%%%%%%%%%%%%%%%%

%% -- The above has original references as well -> check page 2/20 LHS for full references

Based on the above described patterns in natural interactions between individuals, this study also investigates the efficacy of Novelty Search as a fitness function in the Evolutionary Algorithm.


One of the biggest issues to overcome in order to make these agents as robust as possible (and more suited to their inherently dangerous environments) is determining some way for the robots to be able to continue performing their required tasks despite any damage to their physical morphology that they may sustain. More specifically, with regards to their sensory input devices, which are integral to the functioning of any agent.

In fact, an open problem in \textit{collective} \cite{KubeZhang1994B} and \textit{swarm robotics} \cite{Beni2004}
is ascertaining appropriate sensory-motor configurations (morphologies) for robot teams that must work collectively given automatically generated controllers \cite{FloreanoDurrMattiussi2008}.

Current robotic systems make use self-diagnosis and selection from pre-designed contingency plans in order to be able to recover from damage and continue its normal functioning \cite{fenton2001fault, verma2004real, BongardZykovLipson2006}. However, robots that implemented this approach of self-diagnosis and recovery proved to be somewhat problematic in that these systems are relatively expensive and are difficult to design since it requires full \textit{a priori} knowledge of the system in order to design the contingency plans \cite{CullyCluneTaraporeMouret2015}.




%%%%%%%%%%%%%%%%%%%%%%%%%%%%%%%%%%%%%%%%%%%%%%%%%%%%%%%%%%%%%%%%%%%%%
%% 					From the AAMAS Extended Abstract 		       %%
%%		Move this part below to the Research Objectives section    %%
%%%%%%%%%%%%%%%%%%%%%%%%%%%%%%%%%%%%%%%%%%%%%%%%%%%%%%%%%%%%%%%%%%%%%

Addressing this, recent work in \textit{Evolutionary Robotics} elucidated the efficacy of population based stochastic trial-and-error methods for online damage recovery in autonomous robots operating in physical environments \cite{CullyCluneTaraporeMouret2015}.
This was demonstrated as being akin to self-adaptation and injury recovery of animals observed in nature.

This study further contributes to this research area, focussing on \textit{evolutionary controller design} \cite{FloreanoDurrMattiussi2008} within the broader context of \textit{collective} \cite{KubeZhang1994B} and \textit{swarm} \cite{Beni2004} robotics.
That is, evolutionary controller design for robot groups that must continue to accomplish tasks given that damage is sustained to the morphologies of some or all of the robots in the group.

%%%%%%%%%%%%%%%%%%%%%%%%%%%%%%%%%%%%%%%%%%%%%%%%%%%%%%%%%%%%%%%%%%%%%%%


Given previous non-objective controller evolution work in collective robotics such as the following;
\cite{gomes2013generic}
\cite{RefWorks:5}
\cite{RefWorks:11},
and previous research demonstrating the efficacy of \textit{novelty search} \cite{lehman2011abandoning} for evolving behaviours that operate across various robotic morphologies, the following research objectives formed the focus of this study;

\begin{enumerate}
	\item Novelty search behaviours out-perform those of objective search in terms of average team task performance in increasingly complexy collective construction tasks.
	\item Novelty search is suitable for evolving morphologically robust controllers in collective construction tasks.
\end{enumerate}

%%%%%%%%%%%%%%%%%%%%%%%%%%%%%%%%%%%%%%%%%%%%%%%%%%%%%%



\subsection{Research Goals}

Specifically, collective robotic systems that must adapt to unforeseen morphological change, such as the loss or damage of sensors on one of more robots without significant task performance degradation \cite{BongardZykovLipson2006} \cite{CullyCluneTaraporeMouret2015}.

The aims of this research project can be summarised as follows:
\begin{itemize}
	\item To investigate the morphological robustness of multi-robot team controllers evolved using the evolutionary algorithm with respect sensory configurations when implemented in solving increasingly complex tasks requiring collective behaviour
	% \item 'building on the first, the second aim is to investigate two different variations of the evolutionary algorithm by implementing  %various fitness functions'
	\item To demonstrate the efficacy of Novelty Search (as compared to objective-based search) for evolving morphologically robust controllers over increasingly complex tasks.
	\item To demonstrate the efficacy of Novelty Search evolved behaviours versus those evolved by objective-based search in terms of average task performance (behaviour quality) over increasing task complexity
\end{itemize}

% For all of the above aims, the HyperNEAT algorithm \cite{StanleyDAmbrosioGauci2009} is used to evolve homogeneous ANN controllers and all team controllers are evaluated according to their performance in the same collective construction task at various levels of complexity.

To test these objectives, experiments evaluated various robot morphologies in an increasing complex collective construction task. 
That is, morphological robustness of evolved controllers was evaluated in terms of a controller's task performance coupled with alternate robot morphologies different from that with which they were evolved originally.

The HyperNEAT algorithm \cite{StanleyDAmbrosioGauci2009} was used to perform the controller evolution processes for all parts of this experiment

The complexity of the collective construction task is controlled by using a construction schema, which is implemented as a set of underlying connection rules. 

The level of cooperation is controlled by using the different types of resource blocks in the simulation.

The various levels are outlined as follows:
\begin{enumerate}
	\item Level 1: no complexity or cooperation.
	\item Level 2: no complexity and some cooperation.
	\item Level 3: some complexity and some cooperation.
\end{enumerate}

This study's research objectives were formulated given results of the following previous related work. First, that non-objective based controller evolution has been effectively demonstrated in collective robotics \cite{RefWorks:11, gomes2013generic, RefWorks:5}.
Second, that Novelty Search has been demonstrated as suitable for evolving controllers (behaviours) that effectively operate across a range of robot morphologies



This study's contribution was thus to elucidate the impact of specific objective versus non-objective search methods on the evolution of \textit{morphologically robust} controllers in collective robotic systems. 

To date, the morphological robustness of such controller evolution approaches has not been comparatively evaluated, especially in the context of collective robotics. 


\subsection{Contributions}

First and foremost, this study aims to show the ability of HyperNEAT to evolve controllers to be able to sustain physical morphological damage within hazardous environments and still be able to coomplete the task at hand. This is particularly advantageous since a large majority of the environments in which robots are intended to operate autonomously are inherently too dangerous for humans.

Showing the efficacy of the HyperNEAT algorithm with regards to being able to produce a robust robot team controller such that it is able to successfully solve complex collective construction tasks given changes to its sensory input, the main consideration being sustaining damage while operating in hazardous environments. On this same train of thought, it can be applied to evolving robust controllers that only need to be trained/evolved once to solve a task, and can then be transferred to other physical robots with different morphologies, depending on the environment in which it is intended to operate. For example, robots intended for underwater construction would inherently require a different sensory input array than a team of robots intended to perform the same complicated task in an extra-terrestrial environment. In situations such as these, it would be beneficial to be able to only need evolve the controller once so as to be able to solve the task and then transfer them, than it would be to have to evolve controllers for each type of robot morphology.


\subsection{Scope}

This dissertation is mainly focussed on exploring the ability of the HyperNEAT algorithm to produce homogenous multi-agent ANN controllers which are robust to changes in their sensory input morphology. That is to say, whether or not a team of evolved controllers/agents are able to still accomplish the task for which they were trained despite having sustained damage to their input sensors in some way.

Due to the large variety of approaches and parameter tuning abilities when it comes to Neuroevolution methods, we have chosen to focus specifically on the following search functions combined with indirect encoding method of NEAT, called HyperNEAT:
\begin{itemize}
	\item Objective Search
	\item Novelty Search
	%\item Hybrid Search
\end{itemize}

Additionally, these experiments will be implemented using a predetermined collection of sensors that are either arranged in different physical locations on the robot or disabled to simulate damage.  The two distinctive facets of the evaluations is that the controllers are re-implemented using different morphologies as well as an additional evaluation where the sensors are randomly activated/deactivated.

\subsection{Structure}
This dissertation has been structured in the following sections
\begin{itemize}
	\item Introduction: A brief outline of the research and related work as well as some of the applications and benefits of this research.
	\item Related Research: An in-depth investigation into the various main topics incorporated in this investigation (Machine Learning, Evolutionary Algorithms/Computing, Multi-Agent tasks).
	\item Methods: An outline of the various tools and approaches that were used in order to be able to conduct this research. This includes the simulator that was to simulate the environment, the machine learning library used to implement the HyperNEAT algorithm, the sensors and robot types that were simulated, as well as different morphologies and calculations for the fitness functions.
	\item Results: The results of the different experiments that were conducted. This includes the graphs and descriptions of what the different metrics mean and how they are interpreted.
	\item Conclusion: A section in which we discuss the results shown in the previous section and how they relate to the research goals and objectives.
\end{itemize}



% \begin{flushright}
% Guide written by ---\\
% Sunil Patel: \href{http://www.sunilpatel.co.uk}{www.sunilpatel.co.uk}\\
% Vel: \href{http://www.LaTeXTemplates.com}{LaTeXTemplates.com}
% \end{flushright}
