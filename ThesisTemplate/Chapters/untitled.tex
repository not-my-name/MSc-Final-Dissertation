% Chapter 1

\chapter{Introduction} % Main chapter title

\label{Chapter1} % For referencing the chapter elsewhere, use \ref{Chapter1}

%----------------------------------------------------------------------------------------

% Define some commands to keep the formatting separated from the content
\newcommand{\keyword}[1]{\textbf{#1}}
\newcommand{\tabhead}[1]{\textbf{#1}}
\newcommand{\code}[1]{\texttt{#1}}
\newcommand{\file}[1]{\texttt{\bfseries#1}}
\newcommand{\option}[1]{\texttt{\itshape#1}}

%----------------------------------------------------------------------------------------

\section{Introduction}

The competition for survival has been around since the start of time, with all the single celled organisms swimming around in primordial mud, some organisms developed a grouping of sensory cells that allowed them to differentiate between light and dark. For a species that survives through photosynthesis, this became an invaluable ability.

As time continued and natural selection took place, with the 'sighted' cells becoming more successful in finding food and thus being able to reproduce and pass its successful genes on to the next generation. 

'Natural selection is the differential survival and reproduction of individuals due to differences in phenotype. It is a key mechanism of evolution, the change in the heritable traits characteristic of a population over generations. Variation exists within all populations of organisms. This occurs partly because random mutations arise in the genomre of an individual organism, and offspring can inherit such mutations. Throughout the lives of the individuals, their genomes interact with their environments to cause variation in traits. The evnironment of a genome includes the molecular biology in the cell, other cells, other individuals, populations, species, as well as the abiotic environment. Because individuals with certain variants of the trait tend to survive and reproduce more than individuals with other, less successful variants, the population evolves. Other factors affecting reproductive success include secual selection and fecundity selection'

This powerful and unstoppable driver of change has produced some of the most incredible adaptations to allow animals to survive in pretty much any type of environment.

can maybe say something about the evolution of the eye? the whole thing about it needing the adapt to see underwater and then had to adapt to see on land

From the humble origins of the single-celled organism, to the great blue whale and the naked mole rat. The evidence as to the problem solving power of mother nature is boundless. In every nook and cranny on the planet we can find some specialised organism that has adapted in such a way as to be the most successful creature to survive in that habitat. Over hundreds and thousands of generations, these creatures have undergone changes on the genetic level, discovering which characteristics are useful for survival and which ones are suboptimal, resulting in the organisms themselves failing to pass on their genes.

An important requirement to the success of any living organism is the way in which it interacts with its environment. The method in which the organism is able to receive input from their various senses, providing them with information related to their immediate surroundings. For example, sharks have developed the ability to detect electrical currents and fields at extremely long distances. This has been aptly named 'Electroreception' and allows them to perceive natural electrical stimuli



It is really difficult to be able to determine the correct sensory input that would be suitable for so many different envioronments. Additionally, we have some evidence showing how difficult it is to evolve these body brain couplings. As such, we are investigating whether it is even necessary to find the most optimal design. We are checking if controllers evolved with the HyperNEAT algorithm are able to successfully complete their task regardless of changes to their sensory input configuration.

Not only are we looking at the characteristics of the animals and organisms, but we are also considering their behaviour when interacting with other members of the same species. For example, wolves have found that hunting in packs and cooperating to take down larger prey has been the more successful approach. Alternatively, there are organisms such as the polar bear, which thrive on working alone.

The comparison being made above specifically refers to the behaviour where multiple smaller predators have started cooperating in order to hunt where as a single massive polar bear (being classified as an apex predator) is able to hunt/fend for themselves without the assistance of any other animals.


The current trend in increasing levels of computerisation coupled with the increasing complexity of tasks that need to be performed has led to a growing demand for the automated production of these problem solvers \cite{RefWorks:33}.

Given the levels of complexity and non-linearity of real-world problems, traditional methods of controller design have become impractical since they mostly rely on linear mathematical models \cite{RefWorks:32}.

One of the central themes of research in mathematics and computer science is the design and development of automated problem solvers \cite{RefWorks:33}.

The problem solving power of evolution is abundantly clear in nature with the diverse range of organisms that exist (or ever have existed) on Earth, each specially designed for optimal survival in its own niche environment \cite{RefWorks:33}, making it unsurprising that computer scientists have resorted to researching natural processes for inspiration. 

"While it may be possible to solve many complex tasks using a single sophisticated robot, there are several advantages to using a multi-robot team instead. Some of these advantages include being able to use much simpler designed robots in the team instead of one very expensive robot as well as having increased fault tolerance since there is no longer just a single point of failure (each robot in the team is a point of failure" (cite the introduction of your honours final paper)

The ability to automate construction tasks through evolving ANN controllers for multi-robot teams using EAs presents several considerable benefits. Automation such as this could be used to assist in projects such as producing low-cost housing and reduce high accident rates due to human error that accompany traditional construction methods \cite{Khoshnevis2003}. It would be useful to send multi-robot construction teams to extraterrestrial planets in order to build habitable structures in anticipation of humans arriving. These robot teams would also be useful in underwater construction which is difficult and often extremely dangerous \cite{RefWorks:30}.


'There is no a priori reason why machine learning must borrow from nature. A field could exist, complete with well-defined algorithms, data structures, and theories of learning, without once referring to organisms, cognitive or genetic structures, and psychological or evolutionary theories. Yet at the end of the day, with the position papers written, the computers plugged in, and the programs debugged, a learning edifice devoid of natural metaphor would lack something. It would ignore the fact that all these creations have become possible only after three billion years of evolution on this planet. It would miss the point that the very ideas of adaptation and learning are concepts invented by the most recent representatives of the species Homo sapens from the careful observation of themselves and life around them. It would miss the point that natural examples of learning and adaptation are treasure troves of robust procedures and structures. Fortunately, the field of machine learning does rely on natures bounty for both inspiration and mechanism. Many machine learning systems now borrow heavily from current thinking in cognitive science, and rekindled interest in neural networks and connectionism is evidence of serious mechanistic and philosophical currents running through the field. Another are where natural example has been trapped is in work on genetic algorithms and genetics-based machine learning. Rooted in the early cybernetics movement, progress has been made in both theory and application to the point where genetics-based systems are finding their way into everyday commercial use' \cite{goldberg1988genetic}.

%\hl{outline the 2 most powerful problem solving tools found in nature: brain and evolution}
Assuming that the human brain is the greatest problem solving tool in nature, then by extension we can argue that the process of evolution is the greatest problem solver since it is the process that discovered the brain.
Which is the best problem solver, the brain which discovered the wheel and all of the technological advances we can see today, or the process that produced the brain.

'Of the two natural archetypes of learning available to us - the brain and evolution - why have genetic algorithm researchers knowingly adopted the "wrong" metaphor?' \cite{goldberg1988genetic} \hl{find some more information on why this is the 'wrong' metaphor? also, find some more information explaining that these are the 2 archetypes for learning}
'One reason is expedience. The processes of natural evolution and natural genetics have been illuminated by a century of enormous progress in biology and molecular biology. in contrast, the brain, though yielding some of its secrets, remains largely and opaque gray box; we can only guess at many of the fundamental mechanisms contained therein.' \cite{goldberg1988genetic}.




\hl{additional complexity inherent in producing organised collective behaviour}

An open problem in \textit{collective} \cite{KubeZhang1994B} and \textit{swarm robotics} \cite{Beni2004}
is ascertaining appropriate sensory-motor configurations (morphologies) for robots comprising teams that must work collectively given automatically generated controllers \cite{FloreanoDurrMattiussi2008}.

The complex nature of the intricate dynamics inherent in producing self-organised behaviour makes it practically impossible to design these multi-robot team controllers by hand \cite{RefWorks:11}. \hl{can maybe reword this somehow? seems a bit lengthy and out of place}

"Evolutionary Computing is a branch of computer science research  that draws its inspiration from the Darwinian process of evolution through natural selection \cite{EibenSmith2003}."

%\hl{mention here the different biological inspirations that were considered. The brain and the process that found the brain. Present some information on how the both of them work}

In this particular case, researchers have looked into using the most powerful problem solving tool that has ever existed, evolution through the process of natural selection. Starting from single-celled organisms, the process of evolution has produced the incredibly diverse range of organisms that we can see on Earth today.

\hl{information on finding the best way to find the best body brain coupling}
In \textit{evolutionary robotics} \cite{DoncieuxBredecheMouretEiben2015},
a popular method is to co-evolve robot behaviors and morphologies
\cite{LipsonPollack2000}, \cite{Lund2003}, \cite{BuasonBergfeldtZiemke2005}, \cite{AuerbachBongard2014}.


\hl{motivations for different encodings}

Within the purview of such \textit{body-brain} co-evolution, indirect encoding (developmental) methods
have been effectively demonstrated for various single-robot tasks \cite{MautnerBelew2000},
\cite{HornbyPollack2002}, \cite{CheneyLipson2013}.

In cases such as the above, the behaviour and the physical structure of the agent are both evolved at the same time. During the evolution processes (parent selection, mutations/cross-overs, reproduction) affect both of these criteria. In this case, both the behaviour (structure of the ANN, where behaviour is the action produced based on a certain input) and the physical structure/morphology (the actual configuration of the agent's sensory input) are both encoded in the genotype and these properties are affected by the evolutionary operators.
\hl{above explanation is in my own words, but should maybe find some references in order to back this up}





'Of course, simple expedience is not the best reason for adopting a particular course of action, and at first glance, it is not at all obvious why learning in natural or artificial minds should be anything like the adaptation that has occurred in evolution. Yet there is an appealing symmetry in the notion that the mechanisms of natural learning may resemble the processes that created the species possessing those learning processes. Furthermore, the idea that the mind is subject to the same competitive-cooperative pressures as evolutionary systems has achieved some currency outside of GA circles ' \cite{goldberg1988genetic}.

'Despite these suggestions, genetic algorithms and genetics-based machine learning have often been attacked on the grounds that natural evolution is simply too slow to accomplish anything useful in an artificial learning system; three billion years is longer than most people care to wait for a solution to a problem. However, this slowness argument ignores the obvious differences in timescale between natural systems and artificial systems.' \cite{goldberg1988genetic}.

ARGUMENT FOR COMPLEXITY ACHIEVED USING EVOLUTIONARY PROCESSES
' A more fundamental fault is that this argument ignores the robust complexity that evolution has achieved in its three billion years of operation. The "genetic programs" of even the simplest living organisms are more complex than the most intricate human designs' \cite{goldberg1988genetic}

\hl{should maybe find the original references mentioned below}
'Waddington (1967) presents more sophisticated probabilistic arguments that actual evolutionary processes have achieved complexity in existing species that is incommensurate with an evolutionary process using only selection and mutation. Although such argmuments were originally meant to challenge evolutionary theory, genetic agorithmists see no such challenge. Insted, the high speed-to-complexity level observed in nature lends support to the notion that reproduction, recombination, and the processing of building blocks result in the rapd development of appropriate complexity. Moreover, this speed is not purchased at the cost of generality. The mechanisms of genetics and genetic algorithms permit relative efficiency across a broad range of problems' \cite{goldberg1988genetic}.
Although the process of evolution may take hundreds and/or thousands of generations to find an appropriate solution, it is able to produce a level complexification that exceeds that of the solutions found by neurocomputation (the brain process).


given the added complexity inherent in this body-brain coonfiguration, we chose to focus this research solely on evolving the brain and instead using a preconfigured set of input sensors and then comparing the resultant task performance. In this study, we implement increasing complexity in the task that needs to be solved in order to test the robustness of the brains that have been evolved with respect to unexpected changes in their input. We are not investigating the algorithms ability to find the best input configuration body-brain coupling, instead we are trying to determine how robust the evolved controller/brain is with respect to unexpected and undesirable changes in its sensory input configuration. We would like to see how the solution of the problem has been 'embedded' in the network/controller itself. If the solution to the task contained in the network is dependent on the type of input that it can receive.

By transferring a controller across several input configuration (morphologies), we can determine if the controller even needs to know what type of input it is receiving. When being transferred to another input, the controller does not know whether it is receiving input from the same type of sensor as they were evolved with. For example, the controller was evolved with an Infra-red sensor at input node 1, but during one of the evaluation trials where it is transferred to another input config, the input at node 1 might be coming from a Camera or an Ultrasonic sensor.

This might be going into too much detail considering how basic the simulator was and how little detail we could really control.

\subsection{Collective Construction Task}


--the below can be part of research contributions?? explaining some related research--
However, with few exceptions \cite{AsaiArita2003}, \cite{WatsonNitschke2015SSCI}, \cite{HewlandNitschke2015}
there has been relatively little work that evaluates developmental methods to evolve behavior and morphology
for collective robotics tasks, excluding related research in self-assembling and
reconfigurable collective robotics \cite{OGradyDorigo2012}, \cite{RubensteinCornejoNagpal2014}.

Specifically, this study focuses on developmental methods to evolve controllers that exhibit consistently
robust behaviors such that robot teams effectively adapt to the loss of sensors or new sensory configurations
without significant degradation of collective task performance, that is, \textit{morphologically robust behavior}.





HyperNEAT was selected as it is a developmental encoding NE method with demonstrated benefits
that include exploiting task geometry to evolve modular and regular ANN controllers with
increased problem-solving capacity \cite{VerbancsicsStanley2011}, \cite{DAmbrosio2013}.





The \textit{Collective construction} task \cite{WerfelPetersenNagpal2014} was selected since it benefits
from fully automated robot teams that must exhibit robust behavior to handle changing task constraints,
potential robot damage and sensor noise.
For example, collective construction of functional structures and habitats in remote or hazardous
environments \cite{WerfelNagpal2008}.

As in related work \cite{WerfelPetersenNagpal2014}, the task was for robots to search the environment
for \textit{building-block} resources, then move them such that are connected to other blocks.
The task is solved if the team connects all blocks forming a structure during its \textit{lifetime}, and this is equated
with optimal task performance.
However, the task is considered partially solved if only a sub-set of the blocks are connected by the team,
though in such instances, team task performance is proportional to the number of blocks connected.
In this study, task complexity was equated with the number of robots needed
to collectively push blocks together (cooperation) to be connected as a structure and whether the blocks
must be connected in a specific sequence, that is, according to a \textit{construction schema},
table \ref{tab:taskComplexity}.
%Task complexity was regulated via the environment containing three block types, where
%one, two and three robots were required to move each of the block types, respectively.

Hence, we report a preliminary investigation into developmental methods (HyperNEAT is tested in this case
study) for evolving ANN controllers that are robust to morphological change in robotic teams or swarms that
must operate in dynamic, noisy and hazardous environments.



[There is probably something that can also be said about how this applies to finding solution controllers that are highly effective in extremely specific tasks, but barely effective at all when it comes to attempting any other task]

Although Reinforcement Learning (RL) techniques can theoretically solve a number of problems without examples of correct behaviour, in practice they scale poorly with problems that have large state spaces or non-Markov tasks, which are tasks where the state of the environment is only partially observable \cite{RefWorks:32}.

[can probably elaborate a bit more on the different types of tasks that can be accomplished here]

Neuroevolution (NE) is a method of artificially evolving Artificial Neural Networks (ANNs) by implementing Evolutionary Algorithms (EAs) to perform tasks such as training the connection weights and designing the network topology\cite{RefWorks:31,RefWorks:1}.

By evolving a population of ANNs, NE searches through the space of possible controllers by evaluating each candidate solution using a fitness function and selecting the most successful individuals for reproduction, a process analogous to natural selection \cite{RefWorks:32,gomez2001neuro}.

This predetermined fitness function must be designed so that its implementation will guide the NE's search through the problem space, such that it indicates the desired behaviour from the ANN.

The different NE methods can be roughly split into two categories according to their encoding method, either direct or indirect. In direct encoding methods, each element in the genotype explicitly encodes an independent aspect of the phenotype, such that there is a one-to-one mapping between the genotype and the phenotype. In indirect (also called generative or developmental) encoding methods, each element in the genotype is implicitly described by a computable function allowing for a much more compact representation of the genotype. It also allows for genetic information to be reused \cite{clune2011performance,Stanley2003}.

The high level of complexity inherent in real life tasks tend to result in deceptive fitness landscapes \cite{RefWorks:11}. These are fitness landscapes of multimodal problems that have numerous local optima \cite{RefWorks:33}. In multimodal problems, it is possible for the NE search to get stuck around a local optima and lead to the NE method prematurely converging to a suboptimal solution \cite{RefWorks:11}.

For this reason, various NE search functions (also referred to as fitness functions) are investigated. The first search function being the objective search which simply evaluates the suitability of a solution by using an objective fitness function \cite{lehman2011abandoning}.
Novelty search is an approach that provides candidate solutions with a reward based how different their behaviour is from other candidate solutions \cite{RefWorks:11}. The third approach is a hybrid search which attempts to combine EAs with a local search \cite{Castillo2007}.

The collective construction task requires that a team of agents cooperate by coordinating their behaviours in order to assemble various structures within their environment \cite{RefWorks:15}.

The collective gathering task requires a team of agents to coordinate sub-tasks amongst themselves in order to produce a collective behaviour that maximises the resources gathered \cite{RefWorks:15}.

It has been shown that NE can successfully be applied to complex control tasks that require collective behaviour. These control tasks have no obvious mapping between the input from sensory configurations and motor outputs \cite{RefWorks:15}



This thesis is therefore focussed on investigating the effectiveness of various NE methods in solving collective behaviour tasks, with specific emphasis on the indirect encoding method called HyperNEAT \cite{stanley2009hypercube}, as well as the aforementioned search functions.


Another reason for using HyperNEAT is its success in evolving team (collective) behaviours for various multi-agent tasks including RoboCup soccer and pursuit evasion \cite{hausknecht2012hyperneat}.


\subsection{Areas of Research in Computer Science}

\subsubsection{Pattern Classification}
'The task of pattern classification is to assign an input pattern (like speech waveform or handwritten symbol) represented by a feature vector to on of many prespecified classes. Well-known applications include character recognition, speech recognition, EEG waveform classification, blood cell classification, and printed circuit board inspection' \cite{jain1996artificial}.

\subsubsection{Clustering/Categorization}
'In clustering, also known as unsupervised pattern classification, there are no training data with known class labels. A clustering algorithm explores the similarity between the patterns and places similar patterns in a cluster. Well-known clustering applications include data mining, data compression, and exploratory data anyalysis ' \cite{jain1996artificial}.

\subsubsection{Function Approximation}
'Suppose a set of $n$ labeled training patterns (input-output pairs), {($x_1$, $y_1$), ($x_2$, $y_2$),..., ($x_n$, $y_n$)} have been generated from an unknown function $f(x)$ (subject to noise). The task of function approximation is to find an estimate, say $f^1$, of the unknown function $f$. Various engineering and scientific modeling problems require function approximation' \cite{jain1996artificial}.

\subsubsection(Prediction/Forecasting)
'Given a set of $n$ samples {$y(t_1)$, $y(t_2)$,..., $y(t_n)$} in a time sequence, $t_1$, $t_2$,..., $t_n$' the task is to predict the sample $y(t_(n+1))$''


\subsection{Possible Applications}
Just a list of all the possible future applications for the research that is being performed in this investigation.

In this section, we go on to explain some possible applications of multi-robot teams that are able to solve the collective construction task, specificall:

\subsubsection{Underwater Construction}
- Underwater construction: This is one of the most dangerous occupations in the entire world. Be sure to give some research related to studying the risks and the need for being able to perform underwater construction. Find some examples like the underwater roads and internet cables
There could be a broader section title like "Construction in Dangerous Areas". This basically applies to

\subsubsection{Space Exploration/Terraforming}

- Terraforming remote planets: Can mention something to do with all of the Mars missions being performed at the moment and how that could be the next frontier considering we are fucking up the Earth so much. This could allow us to jumpstart the terraforming process without needing to figure out how to get humans+resources+equipment on the planet. We can send a team of robots ahead while we figure things out here. We can send them up to a planet several years ahead of us. These robots will arrive and begin to explore the planet, figuring out which resources are available and how they can be used to build any sort of structural habitat that would be safe for humans to arrive at in the future.

\subsubsection{Production of Low Cost Housing}

especially for remote rural areas where it is difficult for traditional construction equipment to gain access or to get construction materials transported there.

- Production of low cost housing in remote and rural areas: This is an important aspect. Try find some examples of people needing to build some sort of housing in dangerous areas where you can't get normal construction equipment in to the area

- Reduce human error: Prevent all damage that can be caused by human error. Loss of human life would also be minimised

- Can maybe expand and take a look at using these types of ANN controllers for robots that can be used in mines for long periods of time. Reduce the loss of human life again

\subsubsection{Example Search and Rescue Teams}
This section could perhaps be moved to the Introduction part of the paper? It is an existing application but it is more an example reference for cooperative collective robotics like the robocup soccer.

These subsections are explaining other possible applications of the collective construction task specifically.

\section{Problem Statement}

Considering the dangers inherent in the environments in which these robots are meant to be implemented, these robot teams need to be able to handle/accommodate for any damage that it is likely to sustain.


The whole point of using these robots is because it is either too dangerous for humans or the location is too remote for conventional means of access. As such, these robots will have to be able to complete their assigned task and be able to either accommodate sustaining damage or be able to work around it since it will not be possible/feasible for humans to be able to perform any repairs.

Additionally, it would be most effective to have these robots continue working despite any damage they have sustained as it would not be too expensive to continue deploying new replacement robots.

--can maybe rephrase the bottom to make it more relevant to the context of the 'Research Aims' section
In this study, the specific focus is on the evolution of morphologically robust behavior
for robotic teams that must accomplish collective behavior tasks.

Given the added computational complexity of co-evolving body-brain couplings (maybe add a reference to the original mention of this back in the Introduction)
for behaviorally and morphologically heterogenous robots,
this study focuses on evolving collective behaviors for \textit{homogenous} teams with a range of fixed morphologies.



--Spitballing ideas
'There are also other ways of compentaring for the loss of a sensor, particularly because the robot team is homogeneous. It could be possible that some of the other agents are able to accommodate for an agent losing some of its input sensors and its abilities becoming restricted. Lets say that one of the agents loses its construction sensor, it should still be able to assist other robots in carrying a resource around (it can follow another robot that has a construction sensor). Although this may require that the individual robots are able to communicate with each other and indicate when they need assistance and in what capacity.'

Another alternative could be to have a separate robot team that is dedicated solely to performing repairs on the robots that are performing the collective construction task. But then you would also need to have a robot that is capable of performing repairs on the 'repair-robot' since it can also sustain damage, and this would go on recursively forever, needing a construction robot for a repair robot for a construction robot for a repair robot etc.

Since the controllers can't be trained from scratch to use the new sensor morphology and the structure of the ANN can not be altered on-the-fly, the existing controller must be robust in such a way as to be able to use its remaining sensors in order to complete the task at hand.

\hl{some possible future work: Can suggest looking into researching the possiblity of having a heterogeneous robot team in which some of the members are designated 'builders' and some are designated 'repairers'. The repairing robots can be very simple and basic and inexpensive and practically disposable. Their only purpose is to repair components on the 'builders' and would not need to be too complex at all (especially compared to the builder robots). Another approach would be to have a garage of sorts, which would be an extremely robust box with robot arms inside. The damaged builders can then enter the repair place which will then perform the necessary replacements etc. This garage will have to be very robust so as to be able to withstand the harsh conditions}

\section{Research Question}

The research contribution is to demonstrate the efficacy of developmental neuro-evolution (HyperNEAT) for
addressing the evolution of morphologically robust controllers in the context of collective robotics.
To address this objective, this study tested and evaluated five different robot team sensory configurations
(morphologies) in company with three collective construction tasks of increasing complexity.  The fittest
\textit{Artificial Neural Network} (ANN) controller evolved for a given morphology was then transferred and
evaluated in each of the other team morphologies. Such transferred controllers were also evaluated in
each of the three tasks.  In all cases, evolved controller efficacy was evaluated in terms of
collective construction task performance yielded by a robot team.

"Which implementation of the HyperNEAT algorithm is able to produce the most robust controllers in such a way that they are able to consistently solve a collective behaviour task of increasing complexity?"

\subsection{Research Aims}

'The aims of this research project can be summarised as follows:'
\begin{itemize}
	\item 'To investigate the behavioural robustness of multi-robot team controllers evolved using the HyperNEAT algorithm with respect sensory configurations when implemented in solvein increasingly complex tasks requiring collective behaviour'
	\item 'building on the first, the second aim is to investigate the three different variations of the HyperNEAT algorithm by implementing various fitness functions'
\end{itemize}

"The complexity of the collective construction task is controlled by using a construction schema, which is implemented as a set of underlying connection rules. The level of cooperation is controlled by using the different types of resource blocks in the simulation.
The various levels are outlined as follows:"
\begin{enumerate}
	\item Level 1: no complexity or cooperation.
	\item Level 2: some complexity and no cooperation.
	\item Level 3: some complexity and some cooperation.
\end{enumerate}




We apply the HyperNEAT \cite{StanleyDAmbrosioGauci2009} neuro-evolution method to evolve
behavior for a range of morphologically homogenous teams that must solve various collective construction tasks.

\subsection{Research Contributions}

The research contribution is to demonstrate the effectiveness of developmental neuro-evolution (HyperNEAT) for addressing the evolution morphologically robust controllers in the context of collective robotics.

In order to address this objective, this study tested and evaluated five different robot team sensory configurations (referred to as morphologies) in company (in conjunction?) with three collective construction tasks presenting increasing levels of task complexity. The fittest \textit{Artificial Neural Network} (ANN) controller evolved for a given morphology was then transferred to each of the other morphologies and its task performance re-evaluated and compared repectively. Such transferred controllers were also evaluated in each of the three collective construction tasks (each one of increasing complexity). Each controller is only implemented in the task level for which it was evolved \hl{make sure that this is true. Was each controller evolved across all of the complexities or was each controller evolved on a per task basis?}
In all cases, evolved controller efficacy was evaluated in terms of the collective construction task performance yielded by a robot team


\section{Existing Applications And Problem Spaces}

In this section, we go on to outline and examine some of the existing applications that implement a similar solution approach along with the problem spaces that are being addressed by the different investigations.

[Be sure to provide a subsection for each approach that would demonstrate an important aspect of EC]
RoboCup Soccer: cooperation and teamwork
Collective Gathering: So similar to the collective construction task, indicating the efficacy

Make sure to try and find existing work where they use NE to co-evolve the different sensory configurations


In this section, we go on to outline the various implementations of Evolutionary Computing approaches in real-world examples. Some of these examples include the following


Similar research includes investigating the ability of NE methods to find a minimal sensory configuration such that a team of homogenous robots are still able to complete a task requiring collective behaviour \cite{WatsonNitschke2015CEC}, as well as an investigation into evolving specialised sensor resolutions in a multi-agent task \cite{NitschkeSchutEiben2010}.


What makes this project different from existing research, is that it is specifically investigating the ability of HyperNEAT evolved controllers to deal with the loss of input sensors and still be able to solve cooperative tasks of increasing levels of complexity.
The behavioural robustness of a controller will be tested by transferring an evolved controller from its original sensory configuration and implement it with each of the other sensory configurations in the simulation. Once this has been done, the controller is deployed into the simulated environment and its task performance is measured relative to its performance when using the morphology that it had been evolved with.

\hl{for each of the sections below outlining existing research, for each example:
-> have a use case description, illustrating the scenarios that they are evaluating
-> What contributions these researchers are making}


\subsection{Emergency Search-and-Rescue}

\cite{KitanoTadokoro1999} for the entire section below
RoboCup Search and Rescue as a secondary domain for RoboCup activities
'some good examples of how these learned behaviours can be transferrable to other problem domains that require similar approaches'
Disaster rescue is one of the most serious social issues which involves very large numbers of heterogenous agents in the hostile environment

Problem domains/use case scenario
At 5:46AM, 17 January 1995, a large earthquake of a moment magnitude 6.9 hit Kobe city, Japan, killing over 6000 people and crushed almost 1/5th of all the housing structures \cite{KitanoTadokoro1999}.
Immediately after the earthquake, over 300 fires were reported and they started to quickly spread over wider areas. Fire fighting attempts were unsuccessful due to disruptions in the water supply. The earthquake had caused significant structural damage to local reservoirs, resulting in the water having leaked out within a few hours.
To make situations even worse, debris from collapsed wooden houses and surrounding trees cluttered roads and other various, turning these would-be firebreaks into veritable combustion pathways. This debris caught fire while it was on the roads and the open space resulted in winds being channelled right over the fire, causing it to spread even faster along these open spaces and have easy access to the rest of the connected infrastructure.
Additionally, this resulted in paramedics and emergency response teams not being able to reach disaster sites along conventional routes. Which also means that even if they were able to get human teams in there, it would be near impossible to transport the heavy equipment that would be needed (firetrucks, jacks and lifts, whatever tools they need to remove debris).
Aerial surveillance was able to provide an overall view, but with very little details as to the situation on the ground level.
The problem with using helicopters an planes is the noise that they create. This makes it very hard for emergency response teams to hear the faint sounds of people trapped underground and calling for help. These faint sounds that the victims make is the only source of information for these teams (at the time anyways, there may be some more advanced technology by now)


\subsection{RoboCup Soccer}
-cooperation between individuals





\subsection{Collective Gathering}

We chose to elaborate on this approach specifically because it is so similar to the construction task in question that we are investigating in this paper.

\subsection{Balancing Pendulum}

This is the classic problem space for controllers produced using Machine Learning approaches. (Not sure if this is necessarily Evolutionary Algorithms)

This is a good example to have because it is easily applied to increasing degrees of complexity
The balancing pendulum problem has been implemented for both a single, double, and even a triple jointed pendulum
[Can maybe find some pictures or GIFs of the controllers in action]

\subsection{Space Antenna}

This is some good research because it shows that what we think to be optimal design might not necessarily be that at all. It is a good example of how evolutionary computing approaches are able to find solutions that we as humans would never even have considered as being options.

Be sure to find the comparison examples: pictures of the antenna that the controller was able to create


Not too sure where to place this section/
Mention the previous research that has been conducted in this field
-> RoboCup Soccer examples: find some research papers
-> double and triple pendulum example: where the robot can balance the pole upright
-> Collective gathering task
-> Find that paper where they created the space antenna in the super weird shape











































\begin{flushright}
Guide written by ---\\
Sunil Patel: \href{http://www.sunilpatel.co.uk}{www.sunilpatel.co.uk}\\
Vel: \href{http://www.LaTeXTemplates.com}{LaTeXTemplates.com}
\end{flushright}
