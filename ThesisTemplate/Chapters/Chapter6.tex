\chapter{Conclusions}
\label{Chapter6}

                                %%%%%%%%%%%%%%%%%%%%
								%% VERBATIM START %%
                                %%      SSCI      %%  
								%%%%%%%%%%%%%%%%%%%%

This research presented a study on the efficacy of HyperNEAT for evolving \textit{morphologically robust}
behaviors for homogenous robot teams that must solve a collective behavior task of increasing complexity.
That is, the average maximum task performance of behaviors evolved for a given team morphology (robot
sensory configuration) that was then transferred to a different team morphology.
Controllers that did not yield degraded task performance when transferred to another morphology
were considered to be morphologically robust.
The objective was to test and evaluate methods that generate morphological robust behaviors,
where varying morphologies emulated sensor damage or intentional changes to the sensory systems
of robotic teams.

Results indicated that HyperNEAT was appropriate for generating morphologically robust
controllers for a collective construction task of increasing complexity.
This task required robots to cooperatively push blocks such
that they connected together to form structures.
Task complexity was regulated by
the number of robots required to push blocks and a construction schema mandating that specific block types
be connected in specific ways.
These results support the notion that developmental neuro-evolution methods, such as HyperNEAT,
are appropriate for controller evolution in robotics applications where robot teams
must adapt during their lifetime to damage or otherwise must dynamically
adapt their sensory configuration to solve new unforseen tasks.